
\documentclass{article}
\usepackage[landscape, a4paper, margin=1cm]{geometry}
\usepackage{multicol}
\usepackage{listings}
\usepackage{xcolor}
\usepackage{titlesec}
\usepackage{fancyvrb}
\usepackage{ctex} % for Chinese titles

\titleformat{\section}{\Large\bfseries}{\thesection}{1em}{}
\pagestyle{empty}

\lstset{
  basicstyle=\ttfamily\scriptsize,
  breaklines=true,
  frame=single,
  columns=fullflexible,
  language=C++,
  keywordstyle=\color{blue},
  commentstyle=\color{gray},
  stringstyle=\color{red},
  numbers=left,
  numberstyle=\tiny,
  numbersep=5pt,
  tabsize=2,
  showstringspaces=false
}

\begin{document}
\begin{multicols}{3}
\section*{FHQ Treap - 范浩强}
\begin{lstlisting}
  mt19937 myRand(chrono::steady_clock::now().time_since_epoch().count());
  
  struct Node
  {
      int val;
      int rd = myRand();
      Node *lf = nullptr;
      Node *rt = nullptr;
      int cnt = 1;
  };
  
  int Cnt(Node *a)
  {
      return a ? a->cnt : 0;
  }
  
  void Update(Node *a)
  {
      a->cnt = Cnt(a->lf) + Cnt(a->rt) + 1;
  }
  
  void Split(Node *a, int pivot, Node *&l, Node *&r)
  {
      if (a == nullptr)
      {
          l = nullptr;
          r = nullptr;
          return;
      }
      if (a->val < pivot)
      {
          l = a;
          Split(a->rt, pivot, a->rt, r);
          Update(a);
          return;
      }
      r = a;
      Split(a->lf, pivot, l, a->lf);
      Update(a);
  }
  
  void SplitRk(Node *a, int pivot, Node *&l, Node *&r)
  {
      if (a == nullptr)
      {
          l = nullptr;
          r = nullptr;
          return;
      }
      if (Cnt(a->lf) < pivot)
      {
          l = a;
          SplitRk(a->rt, pivot - Cnt(a->lf) - 1, a->rt, r);
          Update(a);
          return;
      }
      r = a;
      SplitRk(a->lf, pivot, l, a->lf);
      Update(a);
  }
  
  Node *Merge(Node *l, Node *r)
  {
      if (l == nullptr)
      {
          return r;
      }
      if (r == nullptr)
      {
          return l;
      }
      if (l->rd < r->rd)
      {
          r->lf = Merge(l, r->lf);
          Update(r);
          return r;
      }
      l->rt = Merge(l->rt, r);
      Update(l);
      return l;
  }
  
  
  
  void Show(Node *a)
  {
      if (a)
      {
          cout << '(';
          Show(a->lf);
          cout << a->val << ',' << a->cnt;
          Show(a->rt);
          cout << ')';
      }
  }
  
  void Insert(int x)
  {
      Node *l, *r;
      Split(root, x, l, r);
      root = Merge(Merge(l, new Node({x})), r);
  }
  
  void Delete(int x)
  {
      Node *l, *mid, *r;
      Split(root, x, l, r);
      SplitRk(r, 1, mid, r);
      delete mid;
      root = Merge(l, r);
  }
  
  int Rank(int x)
  {
      Node *l, *r;
      Split(root, x, l, r);
      int res = Cnt(l) + 1;
      root = Merge(l, r);
      return res;
  }
  
  int Get(int x)
  {
      Node *l, *mid, *r;
      SplitRk(root, x - 1, l, r);
      SplitRk(r, 1, mid, r);
      int res = mid->val;
      root = Merge(Merge(l, mid), r);
      return res;
  }
  
  int Prev(int x)
  {
      Node *l, *mid, *r;
      Split(root, x, l, r);
      SplitRk(l, Cnt(l) - 1, l, mid);
      int res = mid->val;
      root = Merge(Merge(l, mid), r);
      return res;
  }
  
  int Next(int x)
  {
      Node *l, *mid, *r;
      Split(root, x + 1, l, r);
      SplitRk(r, 1, mid, r);
      int res = mid->val;
      root = Merge(Merge(l, mid), r);
      return res;
  }
  
  Node *root;
  
  int n;
  int m;
\end{lstlisting}

\section*{Cost Flow - 费用流}
\begin{lstlisting}
  const int INF = 0x3f3f3f3f;
  
  const int MAXN = 8000;
  const int MAXM = 60000;
  
  struct Edge
  {
      int u;
      ll cap;
      ll w;
      int nxt;
  };
  
  Edge edges[MAXM];
  int cc = 2; 
  int first[MAXN];
  int dqh[MAXN]; // 当前弧
  
  void Add(int x, int y, int cap, int w)
  {
      edges[cc].u = y;
      edges[cc].cap = cap;
      edges[cc].w = w;
      edges[cc].nxt = first[x];
      first[x] = cc;
      cc++;
  }
  
  void AddEdge(int x, int y, int flow, int w)
  {
      Add(x, y, flow, w);
      Add(y, x, 0, -w);
  }
  
  int n;
  int m;
  int p[60][60];
  int w[60][60];
  const int S = 1;
  const int T = 2;
  int N;
  
  int room[60][60];
  int heng[60][60];
  int shu[60][60];
  
  ll h[MAXN];
  long long cost = 0;
  ll flow = 0;
  ll dis[MAXN];
  
  bool SPFA()
  {
      memset(h, 0x3f, sizeof(h));
      h[S] = 0;
      queue<pair<ll, int>> que;
      que.push({0, S});
      while (que.size())
      {
          auto [d, x] = que.front();
          que.pop();
          if (d == h[x])
          {
              for (int i = first[x]; i; i = edges[i].nxt)
              {
                  int y = edges[i].u;
                  if (edges[i].cap && d + edges[i].w < h[y])
                  {
                      h[y] = d + edges[i].w;
                      que.push({h[y], y});
                  }
              }
          }
      }
      return h[T] < INF;
  }
  
  void Flow(int x, ll flow)
  {
      edges[x].cap -= flow;
      edges[x ^ 1].cap += flow;
  }
  
  bool vis[MAXN];
  
  ll Dfs(int x, ll flow)
  {
      if (x == T)
      {
          return flow;
      }
      vis[x] = true;
      ll curFlow = 0;
      for (auto &i = dqh[x]; i; i = edges[i].nxt)
      {
          int y = edges[i].u;
          if (!vis[y] && h[y] == h[x] + edges[i].w && edges[i].cap)
          {
              ll f = Dfs(y, min(flow - curFlow, edges[i].cap));
              if (f)
              {
                  Flow(i, f);
                  curFlow += f;
                  if (curFlow == flow)
                  {
                      vis[x] = false;
                      return flow;
                  }
              }
          }
      }
      vis[x] = false;
      return curFlow;
  }
  
  void Dinic()
  {
      while (SPFA())
      {
          for (int i = 1; i <= N; i++)
          {
              dqh[i] = first[i];
          }
          ll curFlow = Dfs(S, INF);
          flow += curFlow;
          cost += (long long)curFlow * h[T];
      }
  }
\end{lstlisting}

\section*{LCT}
\begin{lstlisting}
  const int MAXN = 100010;
  
  int n;
  
  namespace LCT
  {
      int fa[MAXN];
      int ch[MAXN][2];
      int sz[MAXN];
      bool tag[MAXN];
  #define ls ch[p][0]
  #define rs ch[p][1]
      void PushUp(int p)
      {
          sz[p] = sz[ls] + sz[rs] + 1;
      }
      void Flip(int p)
      {
          tag[p] ^= true;
          swap(ls, rs);
      }
      void PushDown(int p)
      {
          if (tag[p])
          {
              if (ls)
                  Flip(ls);
              if (rs)
                  Flip(rs);
              tag[p] = false;
          }
      }
      void PushAll(int p)
      {
          if (tag[p])
          {
              if (ls)
                  Flip(ls);
              if (rs)
                  Flip(rs);
              PushAll(ls);
              PushAll(rs);
              tag[p] = false;
          }
      }
      void Show()
      {
          for (int j = 1; j <= n; j++)
          {
              cout << "fa " << j << " = " << LCT::fa[j] << ", ";
              cout << "ls " << j << " = " << ch[j][0] << ", ";
              cout << "rs " << j << " = " << ch[j][1] << ", ";
              cout << "tag " << j << " = " << tag[j] << ", ";
              cout << "sz " << j << " = " << sz[j] << endl;
          }
          cout << endl;
      }
  #define Get(x) (ch[fa[x]][1] == x)
  #define IsRoot(x) (ch[fa[x]][0] != x && ch[fa[x]][1] != x)
      void Rotate(int x)
      {
          int y = fa[x];
          int z = fa[y];
          bool k = Get(x);
          if (!IsRoot(y))
          {
              ch[z][ch[z][1] == y] = x;
          }
          ch[y][k] = ch[x][!k];
          fa[ch[x][!k]] = y;
          ch[x][!k] = y;
          fa[y] = x;
          fa[x] = z;
          PushUp(y);
          PushUp(x);
      }
      void Update(int p)
      {
          if (!IsRoot(p))
          {
              Update(fa[p]);
          }
          PushDown(p);
      }
      void Splay(int x)
      {
          Update(x);
          for (int f; f = fa[x], !IsRoot(x); Rotate(x))
          {
              if (!IsRoot(f))
              {
                  Rotate(Get(f) == Get(x) ? f : x);
              }
          }
      }
      int Access(int x)
      {
          int p;
          for (p = 0; x; p = x, x = fa[x])
          {
              Splay(x);
              ch[x][1] = p;
              PushUp(x);
          }
          return p;
      }
      void MakeRoot(int p)
      {
          p = Access(p);
          Flip(p);
      }
      void Link(int x, int y)
      {
          MakeRoot(x);
          Splay(x);
          fa[x] = y;
      }
      int Dis(int x, int y)
      {
          MakeRoot(x);
          Access(y);
          Splay(y);
          return sz[y] - 1;
      }
  }
  
  struct Diameter
  {
      int x;
      int y;
      int len;
  };
  
  bool Cmp(const Diameter &a, const Diameter &b)
  {
      return a.len < b.len;
  }
  
  int p[MAXN];
  Diameter diameters[MAXN];
  
  int curRes = 0;
  
  int P(int a)
  {
      if (a == p[a])
      {
          return a;
      }
      return p[a] = P(p[a]);
  }
  
  pair<int, int> Farthest(int pa, int a)
  {
      auto [x, y, d] = diameters[pa];
      int disx = LCT::Dis(x, a);
      int disy = LCT::Dis(y, a);
      if (disx < disy)
      {
          return {y, disy};
      }
      return {x, disx};
  }
  
  void Init()
  {
      for (int i = 1; i <= n; i++)
      {
          p[i] = i;
          diameters[i] = {i, i, 0};
          LCT::sz[i] = 1;
      }
  }
  
  void Link(int a, int b)
  {
      int pa = P(a);
      int pb = P(b);
      if (diameters[a].len < diameters[b].len)
      {
          swap(pa, pb);
          swap(a, b);
      }
      auto [fara, disa] = Farthest(pa, a);
      auto [farb, disb] = Farthest(pb, b);
      Diameter newD = {fara, farb, disa + disb + 1};
      curRes -= diameters[pa].len + diameters[pb].len;
      diameters[pa] = max({diameters[pa], diameters[pb], newD}, Cmp);
      curRes += diameters[pa].len;
      p[pb] = pa;
      LCT::Link(b, a);
  }
\end{lstlisting}

\section*{SA 后缀数组}
\begin{lstlisting}
  #include <iostream>
  #include <vector>
  
  using namespace std;
  
  struct SA {
      string s;
      int n;
      vector<int> rk;
      vector<int> oldrk;
      vector<int> sa;
      vector<int> id;
      vector<int> cnt;
      vector<int> height;
  
      bool Same(int a, int b, int w) {
          return a < n - w && b < n - w && oldrk[a] == oldrk[b] &&
              oldrk[a + w] == oldrk[b + w] ||
              a >= n - w && b >= n - w && oldrk[a] == oldrk[b];
      }
  
      SA(const string& s, int m) : s(s) {
          n = s.size();
          int p;
          rk.resize(n);
          oldrk.resize(n);
          sa.resize(n);
          id.resize(n);
          cnt.resize(m);
          height.resize(n);
          for (int i = 0; i < n; i++) {
              rk[i] = s[i];
              cnt[rk[i]]++;
          }
          for (int i = 1; i < m; i++) {
              cnt[i] += cnt[i - 1];
          }
          for (int i = n - 1; i > -1; i--) {
              sa[--cnt[rk[i]]] = i;
          }
          for (int w = 1;; w <<= 1, m = p) {
              int cur = 0;
              for (int i = n - w; i < n; i++) {
                  id[cur] = i;
                  cur++;
              }
              for (int i = 0; i < n; i++) {
                  if (sa[i] >= w) {
                      id[cur] = sa[i] - w;
                      cur++;
                  }
              }
              cnt.assign(m, 0);
              for (int i = 0; i < n; i++) {
                  cnt[rk[i]]++;
              }
              for (int i = 1; i < m; i++) {
                  cnt[i] += cnt[i - 1];
              }
              for (int i = n - 1; i > -1; i--) {
                  sa[--cnt[rk[id[i]]]] = id[i];
              }
              p = 0;
              swap(rk, oldrk);
              for (int i = 0; i < n; i++) {
                  if (i && !Same(sa[i - 1], sa[i], w)) {
                      p++;
                  }
                  rk[sa[i]] = p;
              }
              p++;
              if (p == n) {
                  break;
              }
          }
          for (int i = 0, k = 0; i < n; i++) {
              if (rk[i]) {
                  if (k) {
                      k--;
                  }
                  int last = sa[rk[i] - 1];
                  while (i + k < n && last + k < n && s[i + k] == s[last + k]) {
                      k++;
                  }
                  height[rk[i]] = k;
              }
          }
      }
  };
  
  int main() {
      ios::sync_with_stdio(false);
      cin.tie(nullptr);
      string s;
      cin >> s;
      SA sa(s, 128);
      for (int i = 0; i < sa.n; i++) {
          cout << sa.sa[i] << ' ';
      }
      cout << '\n';
      for (int i = 0; i < sa.n; i++) {
          cout << sa.height[i] << ' ';
      }
      return 0;
  }
\end{lstlisting}

\section*{Lyndon 分解}
\begin{lstlisting}
  #include <iostream>
  
  using namespace std;
  
  int res;
  string s;
  int n;
  
  int main() {
      ios::sync_with_stdio(false);
      cin.tie(0);
      cin >> s;
      n = s.length();
      int i = 0, j = 0, k = 1;
      while (i < n) {
          while (k < n && s[k] >= s[j]) {
              if (s[k] == s[j]) {
                  j++;
              }
              else {
                  j = i;
              }
              k++;
          }
          while (i <= j) {
              i = min(n, i + k - j);
              // cout << i << ',' << j << ',' << k << endl;
              res ^= i;
          }
          // cout << i << ',' << j << ',' << k << endl;
          j = i;
          k = i + 1;
          continue;
      }
      cout << res;
      return 0;
  }
\end{lstlisting}

\section*{SAM 后缀自动机
}
\begin{lstlisting}
  class Solution {
      struct Node {
          array<int, 26> nxt;
          int len = 0;
          int link = -1;
          Node() { nxt.fill(-1); }
      };
      vector<Node> nodes;
      int last;
      void Init() {
          nodes = {{}};
          last = 0;
      }
      void Insert(int c) {
          int cur = nodes.size();
          nodes.emplace_back();
          nodes[cur].len = nodes[last].len + 1;
          int p = last;
          while (p != -1 && nodes[p].nxt[c] == -1) {
              nodes[p].nxt[c] = cur;
              p = nodes[p].link;
          }
          if (p == -1) {
              nodes[cur].link = 0;
          } else {
              int q = nodes[p].nxt[c];
              if (nodes[q].len == nodes[p].len + 1) {
                  nodes[cur].link = q;
              } else {
                  int clone = nodes.size();
                  nodes.push_back(nodes[q]);
                  nodes[clone].len = nodes[p].len + 1;
                  nodes[cur].link = clone;
                  nodes[q].link = clone;
                  while (p != -1 && nodes[p].nxt[c] == q) {
                      nodes[p].nxt[c] = clone;
                      p = nodes[p].link;
                  }
              }
          }
          last = cur;
      }
      bool Match(const string& s) {
          int cur = 0;
          for (const auto& c : s) {
              cur = nodes[cur].nxt[c - 'a'];
              if (cur == -1) {
                  return false;
              }
          }
          return true;
      }
  
  public:
      int numOfStrings(vector<string>& patterns, const string& word) {
          Init();
          for (const auto& c : word) {
              Insert(c - 'a');
          }
          int res = 0;
          for (const auto& s : patterns) {
              res += Match(s);
          }
          return res;
      }
  };
\end{lstlisting}

\section*{PAM 回文自动机
}
\begin{lstlisting}
  #include <iostream>
  #include <unordered_map>
  
  using namespace std;
  
  struct Node {
      int len = 0;
      Node* fail = nullptr;
      unordered_map<char, Node*> nxt;
      int cnt = 0;
  };
  
  string s;
  
  int k = 0;
  Node* ji = new Node({ -1 });
  Node* ou = new Node({ 0, ji });
  Node* last = ou;
  Node* cur = ji;
  
  int main() {
      ios::sync_with_stdio(false);
      cin.tie(nullptr);
      cin >> s;
      for (int i = 0; i < s.length(); i++) {
          char c = (s[i] - 97 + k) % 26 + 97;
          s[i] = c;
          while (i - last->len <= 0 || s[i - last->len - 1] != c) {
              last = last->fail;
          }
          auto it = last->nxt.find(c);
          if (it == last->nxt.end()) {
              it = last->nxt.insert({ c, new Node({last->len + 2, nullptr, {}, 0}) }).first;
          }
          while (i - cur->len <= 0 || s[i - cur->len - 1] != c) {
              cur = cur->fail;
          }
          if (cur == last) {
              cur = cur->fail;
              if (cur) {
                  while (i - cur->len <= 0 || s[i - cur->len - 1] != c) {
                      cur = cur->fail;
                  }
              }
          }
          cur = cur ? cur->nxt[c] : ou;
          last = it->second;
          last->fail = cur;
          k = last->cnt = cur->cnt + 1;
          cout << k << ' ';
      }
      return 0;
  }
\end{lstlisting}

\section*{NTT
}
\begin{lstlisting}
  void Init() {
      while (N <= n + m) {
          N <<= 1;
      }
      rev.resize(N);
      F.resize(N);
      G.resize(N);
      FG.resize(N);
      for (int i = 1; i < N; i++) {
          rev[i] = rev[i >> 1] >> 1 | (i & 1) * (N >> 1);
      }
  }
  
  void NTT(vector<int>& f, int d) {
      ll wn;
      ll w;
      int k;
      ll p, q;
      for (int i = 1; i < N; i++) {
          if (i < rev[i]) {
              swap(f[i], f[rev[i]]);
          }
      }
      for (int len = 2; len <= N; len <<= 1) {
          k = len >> 1;
          wn = Pow(YG, (MOD - 1) / len);
          for (int i = 0; i < N; i += len) {
              w = 1;
              for (int j = i; j < i + k; j++) {
                  p = f[j];
                  q = f[j + k] * w % MOD;
                  f[j] = p + q;
                  if (f[j] >= MOD) {
                      f[j] -= MOD;
                  }
                  f[j + k] = p - q;
                  if (f[j + k] < 0) {
                      f[j + k] += MOD;
                  }
                  w = w * wn % MOD;
              }
          }
      }
      if (d == -1) {
          reverse(f.begin() + 1, f.end());
          ll inv = Inv(N);
          for (int i = 0; i < N; i++) {
              f[i] = f[i] * inv % MOD;
          }
      }
  }
\end{lstlisting}

\section*{牛顿迭代

$x_{i+1} = x_i - \frac{f(x_i)}{f'(x_i)}$

## 多项式求逆等
}
\begin{lstlisting}
  #include <iostream>
  #include <vector>
  #include <cstring>
  #include <algorithm>
  #include <bit>
  
  #define ll long long
  
  using namespace std;
  
  const int MOD = 998244353;
  const int YG = 3;
  
  const int N = 1 << 18;
  
  ll Pow(ll a, int b) {
      ll res = 1;
      while (b) {
          if (b & 1) {
              res = res * a % MOD;
          }
          a = a * a % MOD;
          b >>= 1;
      }
      return res;
  }
  
  ll Inv(ll a) {
      return Pow(a, MOD - 2);
  }
  
  int n;
  
  int rev[N];
  int F[N];
  int G[N];
  int inv[N];
  
  void InitMain() {
      inv[1] = 1;
      for (int i = 2; i < N; i++) {
          inv[i] = (ll)(MOD - MOD / i) * inv[MOD % i] % MOD;
      }
  }
  
  void Init(int lim) {
      for (int i = 1; i < lim; i++) {
          rev[i] = rev[i >> 1] >> 1 | (i & 1) * (lim >> 1);
      }
  }
  
  void NTT(int* f, int d, int lim) {
      ll wn;
      ll w;
      int k;
      ll p, q;
      for (int i = 1; i < lim; i++) {
          if (i < rev[i]) {
              swap(f[i], f[rev[i]]);
          }
      }
      for (int len = 2; len <= lim; len <<= 1) {
          k = len >> 1;
          wn = Pow(YG, (MOD - 1) / len);
          for (int i = 0; i < lim; i += len) {
              w = 1;
              for (int j = i; j < i + k; j++) {
                  p = f[j];
                  q = f[j + k] * w % MOD;
                  f[j] = p + q;
                  if (f[j] >= MOD) {
                      f[j] -= MOD;
                  }
                  f[j + k] = p - q;
                  if (f[j + k] < 0) {
                      f[j + k] += MOD;
                  }
                  w = w * wn % MOD;
              }
          }
      }
      if (d == -1) {
          reverse(f + 1, f + lim);
          ll inv = Inv(lim);
          for (int i = 0; i < lim; i++) {
              f[i] = f[i] * inv % MOD;
          }
      }
  }
  
  // fg = 1
  // fg - 1 = 0
  // f - 1 / g = 0
  // g = g - (f - 1 / g) / (1 / g ^ 2) = 2g - fg ^ 2 = g(2 - fg)
  
  int cur1[N];
  int cur2[N];
  
  void Inv(int* f, int* g, int lim) {
      g[0] = Inv(f[0]);
      for (int len = 2; len <= lim; len <<= 1) {
          int k = len << 1;
          int s = len >> 1;
          memcpy(cur1, f, len * sizeof(int));
          memset(cur1 + len, 0, len * sizeof(int));
          memcpy(cur2, g, s * sizeof(int));
          memset(cur2 + s, 0, (k - s) * sizeof(int));
          Init(k);
          NTT(cur1, 1, k);
          NTT(cur2, 1, k);
          for (int i = 0; i < k; i++) {
              g[i] = cur2[i] * (2 + MOD - (ll)cur1[i] * cur2[i] % MOD) % MOD;
          }
          NTT(g, -1, k);
      }
  }
  
  // g ^ 2 = f
  // g ^ 2 - f = 0
  // g = g - (g ^ 2 - f) / 2g = (g ^ 2 + f) / 2g
  
  int cur3[N];
  
  void Sqrt(int* f, int* g, int lim) {
      g[0] = 1;
      for (int len = 2; len <= lim; len <<= 1) {
          int k = len << 1;
          int s = len >> 1;
          memset(g + s, 0, s * sizeof(int));
          Inv(g, cur3, len);
          memset(cur3 + len, 0, len * sizeof(int));
          memcpy(cur1, f, len * sizeof(int));
          memset(cur1 + len, 0, len * sizeof(int));
          memcpy(cur2, g, s * sizeof(int));
          memset(cur2 + s, 0, (k - s) * sizeof(int));
          Init(k);
          NTT(cur1, 1, k);
          NTT(cur2, 1, k);
          NTT(cur3, 1, k);
          for (int i = 0; i < k; i++) {
              g[i] = ((ll)cur2[i] * cur2[i] + cur1[i]) % MOD * ((MOD + 1) / 2) % MOD * cur3[i] % MOD;
          }
          NTT(g, -1, k);
      }
  }
  
  void Derivative(int* f, int* g, int lim) {
      for (int i = 1; i < lim; i++) {
          g[i - 1] = (ll)f[i] * i % MOD;
      }
      g[lim - 1] = 0;
  }
  
  void Integrate(int* f, int* g, int lim) {
      g[0] = 0;
      for (int i = 1; i < lim; i++) {
          g[i] = (ll)f[i - 1] * inv[i] % MOD;
      }
  }
  
  // g = ln(f)
  // dg = df / f
  // g = int(df / f)
  
  void Ln(int* f, int* g, int lim) {
      int k = lim << 1;
      Inv(f, cur3, lim);
      memset(cur3 + lim, 0, lim * sizeof(int));
      Derivative(f, cur1, lim);
      memset(cur1 + lim, 0, lim * sizeof(int));
      Init(k);
      NTT(cur3, 1, k);
      NTT(cur1, 1, k);
      for (int i = 0; i < k; i++) {
          cur1[i] = (ll)cur1[i] * cur3[i] % MOD;
      }
      NTT(cur1, -1, k);
      Integrate(cur1, g, lim);
  }
  
  // g = e ^ f
  // lng - f = 0
  // g = g - (lng - f) / (1 / g) = g(1 - lng + f)
  
  void Exp(int* f, int* g, int lim) {
      g[0] = 1;
      for (int len = 2; len <= lim; len <<= 1) {
          int k = len << 1;
          int s = len >> 1;
          memset(g + s, 0, s * sizeof(int));
          Ln(g, cur3, len);
          memset(cur3 + len, 0, len * sizeof(int));
          memcpy(cur1, f, len * sizeof(int));
          memset(cur1 + len, 0, len * sizeof(int));
          memcpy(cur2, g, s * sizeof(int));
          memset(cur2 + s, 0, (k - s) * sizeof(int));
          Init(k);
          NTT(cur1, 1, k);
          NTT(cur2, 1, k);
          NTT(cur3, 1, k);
          for (int i = 0; i < k; i++) {
              g[i] = (ll)cur2[i] * (MOD + 1 - cur3[i] + cur1[i]) % MOD;
          }
          NTT(g, -1, k);
      }
  }
  
  void Solve() {
      cin >> n;
      int lim = bit_ceil((unsigned)n);
      for (int i = 0; i < n; i++) {
          cin >> F[i];
      }
      Exp(F, G, lim);
      for (int i = 0; i < n; i++) {
          cout << G[i] << ' ';
      }
  }
  
  int main() {
      ios::sync_with_stdio(false);
      cin.tie(nullptr);
      InitMain();
      Solve();
      return 0;
  }
\end{lstlisting}


\end{multicols}
\end{document}
